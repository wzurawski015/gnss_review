% projekt/rozdzialy/qam.tex
% project/chapters/qam.tex
\section{Wzór Quadrature Amplitude Modulation}
Wzór \textbf{modulacji QAM (Quadrature Amplitude Modulation)}, można wyrazić jako:
\[
s_{T(i)}(t) = P_{T} \sum_{u=-\infty}^{\infty} \left[ d_I(i)(u) \cdot g_I(t - uT_b) + i \cdot d_Q(i)(u) \cdot g_Q(t - uT_b) \right]
\]

Gdzie:
\begin{itemize}
    \item \( P_{T} \) – Współczynnik skalujący dla sygnału (często związany z mocą sygnału).
    \item \( d_I(i)(u) \) – Ciąg symboli binarnych dla składowej in-phase (I) w \( u \)-tym symbolu.
    \item \( d_Q(i)(u) \) – Ciąg symboli binarnych dla składowej kwadratury (Q) w \( u \)-tym symbolu.
    \item \( g_I(t) \) – Funkcja odpowiadająca za kształt składowej in-phase (I), często nazywana funkcją odpowiedzi impulsowej.
    \item \( g_Q(t) \) – Funkcja odpowiadająca za kształt składowej kwadratury (Q), również nazywana funkcją odpowiedzi impulsowej.
    \item \( T_b \) – Okres symbolu (czas trwania jednego symbolu).
    \item \( i \) – Indeks kanału lub nośnej (w kontekście wielu nośnych w systemach GNSS).
\end{itemize}

\subsection{Wyprowadzenie Wzoru QAM}
W modulacji QAM, sygnał jest reprezentowany jako kombinacja dwóch sygnałów nośnych o tej samej częstotliwości, ale przesuniętych w fazie o 90 stopni (kwadraturowo). Sygnalizację można opisać wzorem:
\[
s(t) = I(t) \cos(2 \pi f_c t) - Q(t) \sin(2 \pi f_c t)
\]
gdzie \( I(t) \) i \( Q(t) \) są odpowiednio składowymi in-phase i kwadraturową, a \( f_c \) jest częstotliwością nośną.

\subsubsection{Transformacja Fouriera}
Analizując sygnał QAM w domenie częstotliwości za pomocą transformacji Fouriera:
\[
S(f) = \mathcal{F}\{s(t)\}
\]
Możemy uzyskać:
\[
S(f) = \frac{1}{2} \left[ I(f - f_c) + I(f + f_c) \right] - \frac{1}{2i} \left[ Q(f - f_c) - Q(f + f_c) \right]
\]

\subsubsection{Demodulacja Sygnałów QAM}
Demodulacja sygnałów QAM polega na ekstrakcji składowych \( I(t) \) i \( Q(t) \) poprzez mnożenie sygnału \( s(t) \) przez \( \cos(2 \pi f_c t) \) i \( \sin(2 \pi f_c t) \), a następnie filtrację dolnoprzepustową:
\[
r_I(t) = s(t) \cos(2 \pi f_c t)
\]
\[
r_Q(t) = s(t) \sin(2 \pi f_c t)
\]
\[
I(t) = \text{LPF}\{ r_I(t) \}
\]
\[
Q(t) = \text{LPF}\{ r_Q(t) \}
\]
