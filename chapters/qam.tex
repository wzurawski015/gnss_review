% projekt/chapters/qam.tex
% project/chapters/qam.tex

\section{Modulacja QAM}

Wzór modulacji QAM (Quadrature Amplitude Modulation) można wyrazić jako:

\[
s(t) = I(t) \cos(2 \pi f_c t) - Q(t) \sin(2 \pi f_c t)
\]

gdzie $I(t)$ i $Q(t)$ są odpowiednio składowymi in-phase i kwadraturową, a $f_c$ jest częstotliwością nośną. Analizując sygnał QAM w domenie częstotliwości za pomocą transformacji Fouriera:
\[
S(f) = \int_{-\infty}^{\infty} s(t) e^{-j 2 \pi f t} dt
\]

\section{Komponenty Sygnału QAM}
\begin{itemize}
    \item $I(t)$ - Składowa in-phase (często nazywana jako I)
    \item $Q(t)$ - Składowa kwadraturowa (często nazywana jako Q)
    \item $P_T$ - Współczynnik skalujący dla sygnału (często związany z mocą sygnału)
    \item $g_I(t)$ - Funkcja odpowiedzi impulsowej dla składowej in-phase
    \item $g_Q(t)$ - Funkcja odpowiedzi impulsowej dla składowej kwadraturowej
\end{itemize}
