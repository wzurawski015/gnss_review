% projekt/rozdzialy/wspomaganie_satelitarne.tex
% project/chapters/satellite_augmentation_systems.tex
\section{Systemy Wspomagania Satelitarnego}

\subsection{EGNOS (European Geostationary Navigation Overlay Service)}
EGNOS to system wspomagania satelitarnego, który dostarcza korekty dla sygnałów GPS i Galileo, zwiększając ich dokładność i niezawodność w Europie. EGNOS wykorzystuje sieć stacji referencyjnych, które monitorują sygnały GNSS i przesyłają poprawki do geostacjonarnych satelitów, które następnie retransmitują je do użytkowników.

\subsection{WAAS (Wide Area Augmentation System)}
WAAS to amerykański system wspomagania satelitarnego, który działa podobnie do EGNOS, dostarczając korekty dla sygnałów GPS w Ameryce Północnej.

\subsection{MSAS (Multi-functional Satellite Augmentation System)}
MSAS to japoński system wspomagania satelitarnego, który dostarcza korekty dla sygnałów GPS w regionie Azji Wschodniej.

\subsection{GAGAN (GPS Aided Geo Augmented Navigation)}
GAGAN to indyjski system wspomagania satelitarnego, który dostarcza korekty dla sygnałów GPS w Indiach i sąsiednich regionach.
