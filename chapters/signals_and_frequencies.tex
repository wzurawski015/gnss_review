% projekt/rozdzialy/sygnaly_i_czestotliwosci.tex
% project/chapters/signals_and_frequencies.tex
\section{Informacje o sygnałach i częstotliwościach GNSS}
\begin{itemize}
    \item \textbf{GPS}
    \begin{itemize}
        \item \textbf{Częstotliwości}: L1 (1575.42 MHz), L2 (1227.60 MHz), L5 (1176.45 MHz)
        \item \textbf{Sygnały}:
        \begin{itemize}
            \item L1: C/A (Coarse/Acquisition) dla cywilnych, P(Y) dla wojskowych
            \item L2: P(Y) dla wojskowych, L2C dla cywilnych
            \item L5: Nowy sygnał dla zastosowań cywilnych, oferujący lepszą precyzję
        \end{itemize}
    \end{itemize}

    \item \textbf{Galileo}
    \begin{itemize}
        \item \textbf{Częstotliwości}: E1 (1575.42 MHz), E5a (1176.45 MHz), E5b (1207.14 MHz), E6 (1278.75 MHz)
        \item \textbf{Sygnały}:
        \begin{itemize}
            \item E1: Open Service (OS), Public Regulated Service (PRS)
            \item E5a i E5b: Open Service (OS), Safety of Life (SoL)
            \item E6: Commercial Service (CS)
        \end{itemize}
    \end{itemize}

    \item \textbf{BeiDou}
    \begin{itemize}
        \item \textbf{Częstotliwości}: B1 (1561.098 MHz), B2 (1207.14 MHz), B3 (1268.52 MHz)
        \item \textbf{Sygnały}:
        \begin{itemize}
            \item B1: Open Service (OS)
            \item B2: Open Service (OS)
            \item B3: Restricted Service (RS)
        \end{itemize}
    \end{itemize}

    \item \textbf{GLONASS}
    \begin{itemize}
        \item \textbf{Częstotliwości}: L1 (1602 MHz + $k \cdot 0.5625$ MHz), L2 (1246 MHz + $k \cdot 0.4375$ MHz), gdzie \( k \) jest numerem kanału
        \item \textbf{Sygnały}:
        \begin{itemize}
            \item L1: Standard Precision (SP) dla cywilnych, High Precision (HP) dla wojskowych
            \item L2: HP dla wojskowych
        \end{itemize}
    \end{itemize}

    \item \textbf{QZSS}
    \begin{itemize}
        \item \textbf{Częstotliwości}: L1 (1575.42 MHz), L2 (1227.60 MHz), L5 (1176.45 MHz), L6 (1278.75 MHz)
        \item \textbf{Sygnały}:
        \begin{itemize}
            \item L1: C/A, SAIF
            \item L2: L2C
            \item L5: L5
            \item L6: LEX
        \end{itemize}
    \end{itemize}

    \item \textbf{IRNSS (NavIC)}
    \begin{itemize}
        \item \textbf{Częstotliwości}: L5 (1176.45 MHz), S (2492.028 MHz)
        \item \textbf{Sygnały}:
        \begin{itemize}
            \item L5: SPS (Standard Positioning Service)
            \item S: SPS
        \end{itemize}
    \end{itemize}

    \item \textbf{EGNOS}
    \begin{itemize}
        \item \textbf{Częstotliwości}: 1575.42 MHz (L1, GPS)
        \item \textbf{Sygnały}: EGNOS zapewnia korekty różnicowe dla sygnałów GPS i Galileo, zwiększając dokładność i niezawodność.
    \end{itemize}

    \item \textbf{WAAS}
    \begin{itemize}
        \item \textbf{Częstotliwości}: 1575.42 MHz (L1, GPS)
        \item \textbf{Sygnały}: WAAS zapewnia korekty różnicowe dla sygnałów GPS, zwiększając dokładność i niezawodność.
    \end{itemize}

    \item \textbf{MSAS}
    \begin{itemize}
        \item \textbf{Częstotliwości}: 1575.42 MHz (L1, GPS)
        \item \textbf{Sygnały}: MSAS zapewnia korekty różnicowe dla sygnałów GPS, zwiększając dokładność i niezawodność.
    \end{itemize}

    \item \textbf{GAGAN}
    \begin{itemize}
        \item \textbf{Częstotliwości}: 1575.42 MHz (L1, GPS), 1176.45 MHz (L5, GPS)
        \item \textbf{Sygnały}: GAGAN zapewnia korekty różnicowe dla sygnałów GPS, zwiększając dokładność i niezawodność na terenie Indii.
    \end{itemize}
\end{itemize}
