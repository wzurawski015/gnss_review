% projekt/rozdzialy/podsumowanie.tex
% project/chapters/conclusion.tex
\section{Zakończenie}
W niniejszym dokumencie przedstawiono kluczowe metody analizy sygnałów GNSS, podkreślając ich znaczenie w różnych zastosowaniach technologicznych. Każda z tych metod wnosi istotny wkład w rozwój nowoczesnych systemów nawigacyjnych i komunikacyjnych. Dodatkowo omówiono techniki korekcji sygnałów GNSS, algorytmy przetwarzania oraz wpływ atmosfery na sygnały, co stanowi kompleksowy przegląd technologii GNSS na poziomie doktoranckim. Szczególna uwaga została poświęcona teorii synchronizacji czasu oraz technologii White Rabbit, które są kluczowe dla zapewnienia precyzyjnej synchronizacji w zaawansowanych aplikacjach GNSS. Dodatkowo, rozważano integrację GNSS z systemami inercjalnymi (INS), co pozwala na poprawę dokładności i niezawodności w różnorodnych warunkach operacyjnych.
