% projekt/rozdzialy/wstep.tex
% project/chapters/introduction.tex
\section{Wstęp}
Globalne systemy nawigacji satelitarnej (GNSS) są fundamentem współczesnych technologii nawigacyjnych, oferując precyzyjne informacje o położeniu, prędkości i czasie. W niniejszym dokumencie przedstawiona zostanie kompleksowa analiza technologii GNSS, w tym opisy systemów GPS, Galileo, BeiDou, GLONASS, QZSS, IRNSS oraz systemów wspomagania satelitarnego takich jak EGNOS, WAAS, MSAS i GAGAN. Ponadto, omówione zostaną techniki korekcji sygnałów, algorytmy przetwarzania oraz wpływ atmosfery na sygnały GNSS, z uwzględnieniem teorii synchronizacji czasu oraz technologii White Rabbit. Na końcu dokumentu dodano szczegółową analizę integracji GNSS z systemami inercjalnymi (INS).
