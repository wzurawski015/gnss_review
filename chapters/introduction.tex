% projekt/chapters/introduction.tex
% project/chapters/introduction.tex

\section{Wstęp}

Globalne systemy nawigacji satelitarnej (GNSS) są fundamentem współczesnych technologii nawigacyjnych, oferując precyzyjne informacje o położeniu, prędkości i czasie. W niniejszym dokumencie przedstawi\-ono różne aspekty technologii GNSS, w tym sygnały i częstotliwości, modulacje QAM, techniki korekcji sygnałów oraz systemy wspomagania satelitarnego takie jak EGNOS, WAAS, MSAS i GAGAN. Ponadto, omówione zostaną techniki korekcji sygnałów, wpływ atmosfery na sygnały GNSS, z uwzględnieniem teorii synchronizacji czasu oraz technologii White Rabbit. Na końcu dokumentu dodano szczegółową analizę integracji GNSS z systemami inercjalnymi (INS), co pozwala na poprawę dokładności i niezawodności nawigacji.
W niniejszej pracy odwołujemy się do literatury [\cite{examplebook}].
