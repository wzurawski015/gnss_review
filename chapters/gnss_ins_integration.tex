% projekt/rozdzialy/integracja_gnss_ins.tex
% project/chapters/gnss_ins_integration.tex
\section{Integracja GNSS z Inertial Navigation Systems (INS)}
Integracja GNSS z Inertial Navigation Systems (INS) pozwala na poprawę dokładności i niezawodności nawigacji poprzez połączenie zalet obu systemów. GNSS dostarcza dokładnych informacji o położeniu w długim okresie czasu, podczas gdy INS zapewnia wysoką dokładność w krótkim okresie, zwłaszcza w warunkach, gdzie sygnały GNSS mogą być zakłócone lub niedostępne.

\subsection{Podstawy Systemów Inercjalnych}
Systemy inercjalne wykorzystują akcelerometry i żyroskopy do mierzenia przyspieszeń i prędkości kątowych, które są następnie integrowane w celu uzyskania pozycji i orientacji. Podstawowe równania dla systemu inercjalnego można zapisać jako:

\[
\mathbf{v}(t) = \mathbf{v}(t_0) + \int_{t_0}^{t} \mathbf{a}(\tau) d\tau
\]
\[
\mathbf{r}(t) = \mathbf{r}(t_0) + \int_{t_0}^{t} \mathbf{v}(\tau) d\tau
\]

gdzie:
\begin{itemize}
    \item \( \mathbf{v}(t) \) - prędkość w chwili \( t \)
    \item \( \mathbf{a}(t) \) - przyspieszenie w chwili \( t \)
    \item \( \mathbf{r}(t) \) - pozycja w chwili \( t \)
    \item \( t_0 \) - czas początkowy
\end{itemize}

\subsection{Fuzja Danych GNSS i INS}
Integracja GNSS i INS polega na fuzji danych z obu systemów w celu uzyskania bardziej dokładnych i niezawodnych informacji o położeniu i orientacji. W praktyce najczęściej stosuje się Filtr Kalmana rozszerzony (EKF) do fuzji danych GNSS i INS. Podstawowe równania EKF są następujące:

\[
\hat{x}_{k|k-1} = f(\hat{x}_{k-1|k-1}, u_{k-1})
\]
\[
P_{k|k-1} = F_{k-1} P_{k-1|k-1} F_{k-1}^T + Q_{k-1}
\]
\[
K_k = P_{k|k-1} H_k^T (H_k P_{k|k-1} H_k^T + R_k)^{-1}
\]
\[
\hat{x}_{k|k} = \hat{x}_{k|k-1} + K_k (z_k - h(\hat{x}_{k|k-1}))
\]
\[
P_{k|k} = (I - K_k H_k) P_{k|k-1}
\]

gdzie:
\begin{itemize}
    \item \( f(\cdot) \) - nieliniowa funkcja przejścia stanu
    \item \( h(\cdot) \) - nieliniowa funkcja obserwacji
    \item \( F_{k-1} \) - macierz Jacobiana funkcji przejścia stanu
    \item \( H_k \) - macierz Jacobiana funkcji obserwacji
\end{itemize}

\subsection{Przykłady Implementacji Algorytmów GNSS i INS}
Poniżej przedstawiono przykładową implementację filtra Kalmana do integracji danych GNSS i INS w Pythonie:

\lstinputlisting[caption=Implementacja Filtra Kalmana do Integracji GNSS i INS w Pythonie]{ekf_gnss_ins.py}
