\section{Integracja GNSS z systemami inercjalnymi (INS)}

Integracja GNSS z Inertial Navigation Systems (INS) pozwala na poprawę dokładności i niezawodności nawigacji poprzez połączenie zalet obu systemów. GNSS zapewnia długoterminową stabilność i globalne pokrycie, podczas gdy INS zapewnia wysoką dokładność w krótkim czasie, zwłaszcza w warunkach, gdzie sygnały GNSS mogą być zakłócone lub niedostępne.

\section{Model matematyczny}
Podstawowe równania kinematyczne systemów GNSS-INS można wyrazić jako:
\[
\mathbf{p}(t) = \mathbf{p}(t_0) + \int_{t_0}^{t} \mathbf{v}(\tau) \, d\tau + \frac{1}{2} \int_{t_0}^{t} \mathbf{a}(\tau) \, (t - t_0)^2 \, d\tau
\]
gdzie:
\begin{itemize}
    \item $\mathbf{p}(t)$ - pozycja w chwili $t$
    \item $\mathbf{v}(t)$ - prędkość w chwili $t$
    \item $\mathbf{a}(t)$ - przyspieszenie w chwili $t$
    \item $t_0$ - czas początkowy
\end{itemize}

\section{Implementacja EKF w Pythonie}
Poniżej przedstawiono przykładową implementację filtra Kalmana do integracji GNSS i INS w Pythonie:

\lstinputlisting[language=Python, caption={Implementacja EKF w Pythonie}, label=lst:ekf]{/home/wz/gnss_doc_tex/gnss_review/code/ekf_gnss_ins.py}
