% gnss_jamming_interference.tex

\section{GNSS Jamming, Spoofing, and Meaconing (Zakłócenia sygnałów GNSS)}

\subsection{Wprowadzenie}

Zakłócenia sygnałów GNSS, takie jak jamming, spoofing i meaconing, stanowią poważne zagrożenie dla systemów nawigacyjnych. Te zakłócenia mogą prowadzić do utraty sygnału, fałszywych odczytów pozycji oraz innych problemów wpływających na niezawodność i dokładność GNSS.

\subsection{Jamming (Zakłócanie)}

Jamming polega na nadawaniu silnych sygnałów na tych samych częstotliwościach co sygnały GNSS, co powoduje zagłuszenie sygnałów satelitarnych. Może to prowadzić do całkowitej utraty sygnału GNSS i uniemożliwić odbiornikowi określenie pozycji. Przykłady metod przeciwdziałania jammingowi obejmują:

\begin{itemize}
    \item Wykorzystanie technik antenowych o wysokiej kierunkowości.
    \item Zastosowanie filtrowania sygnałów.
    \item Implementacja algorytmów detekcji i unikania zakłóceń.
\end{itemize}

\subsection{Spoofing (Fałszowanie)}

Spoofing polega na wysyłaniu fałszywych sygnałów GNSS w celu wprowadzenia odbiornika w błąd. Atakujący generuje sygnały o wyższej mocy niż sygnały rzeczywiste, co powoduje, że odbiornik odbiera fałszywe dane. Skutki spoofingu mogą obejmować:

\begin{itemize}
    \item Fałszywe odczyty pozycji.
    \item Wprowadzenie użytkowników w błąd.
    \item Możliwość przejęcia kontroli nad autonomicznymi systemami.
\end{itemize}

Przykłady metod przeciwdziałania spoofingowi obejmują:

\begin{itemize}
    \item Wykorzystanie technik autentykacji sygnałów.
    \item Monitorowanie anomalii w danych GNSS.
    \item Zastosowanie wieloczęstotliwościowych odbiorników do porównywania sygnałów z różnych pasm.
\end{itemize}

\subsection{Meaconing (Przekazywanie)}

Meaconing polega na przechwytywaniu sygnałów GNSS i retransmitowaniu ich w innym miejscu, co powoduje, że odbiornik interpretuje fałszywą pozycję jako rzeczywistą. Metody przeciwdziałania meaconingowi są podobne do tych stosowanych w przypadku spoofingu.

\subsection{Środki Zaradcze}

Skuteczne środki zaradcze przeciwko zakłóceniom GNSS obejmują kombinację technik sprzętowych i programowych. Przykłady to:

\begin{itemize}
    \item Implementacja zaawansowanych algorytmów filtrowania i detekcji zakłóceń.
    \item Zastosowanie anten adaptacyjnych.
    \item Integracja z systemami inercyjnymi (INS) i innymi czujnikami wspomagającymi nawigację.
\end{itemize}

\subsection{Podsumowanie}

Zakłócenia sygnałów GNSS, takie jak jamming, spoofing i meaconing, mogą mieć poważne konsekwencje w nawigacji. Środki zaradcze obejmują stosowanie technik autentykacji sygnałów, monitorowanie anomalii w danych GNSS oraz wykorzystanie wieloczęstotliwościowych odbiorników do porównywania sygnałów z różnych pasm. W przyszłości rozwój zaawansowanych technologii przeciwdziałania zakłóceniom GNSS będzie kluczowy dla zapewnienia bezpieczeństwa i niezawodności systemów nawigacyjnych. Opisane m.in. w \cite{maritime_global_security}.

