\chapter{Implementacja EKF w Pythonie}

Poniżej przedstawiono przykładową implementację filtra Kalmana do integracji GNSS i INS w Pythonie:

\begin{lstlisting}[caption={Implementacja EKF w Pythonie}, label={lst:EKF_Python}]
import numpy as np

class EKF:
    def __init__(self, f, h, F, H, Q, R, P, x0):
        self.f = f      # Funkcja przejścia stanu
        self.h = h      # Funkcja obserwacji
        self.F = F      # Jakobian funkcji przejścia stanu
        self.H = H      # Jakobian funkcji obserwacji
        self.Q = Q      # Macierz kowariancji szumu procesowego
        self.R = R      # Macierz kowariancji szumu obserwacji
        self.P = P      # Macierz kowariancji błędu estymacji
        self.x = x0     # Estymata stanu

    def predict(self, u):
        # Predykcja stanu i kowariancji stanu
        self.x = self.f(self.x, u)
        self.P = self.F(self.x, u) @ self.P @ self.F(self.x, u).T + self.Q

    def update(self, z):
        # Obliczanie wzmocnienia Kalmana
        y = z - self.h(self.x)
        S = self.H(self.x) @ self.P @ self.H(self.x).T + self.R
        K = self.P @ self.H(self.x).T @ np.linalg.inv(S)

        # Aktualizacja estymaty stanu i kowariancji błędu estymacji
        self.x = self.x + K @ y
        self.P = self.P - K @ self.H(self.x) @ self.P
\end{lstlisting}
