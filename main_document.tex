\documentclass[12pt]{article}
\usepackage[utf8]{inputenc}
\usepackage[T1]{fontenc}
\usepackage{amsmath}
\usepackage{amsfonts}
\usepackage{amssymb}
\usepackage{geometry}
\usepackage{longtable}
\usepackage{graphicx}
\usepackage{listings}
\usepackage{xcolor}
\usepackage{hyperref}
\usepackage{array}
\usepackage[backend=bibtex,style=numeric]{biblatex}
\addbibresource{bibliography.bib}
\geometry{a4paper, margin=1in}
\sloppy % Dla naprawy underfull hbox

\title{Kompleksowa Analiza Technologii GNSS}
\author{Twoje Imię}
\date{\today}

\begin{document}
\maketitle

\tableofcontents

% projekt/chapters/introduction.tex
% project/chapters/introduction.tex

\section{Wstęp}

Globalne systemy nawigacji satelitarnej (GNSS) są fundamentem współczesnych technologii nawigacyjnych, oferując precyzyjne informacje o położeniu, prędkości i czasie. W niniejszym dokumencie przedstawi\-ono różne aspekty technologii GNSS, w tym sygnały i częstotliwości, modulacje QAM, techniki korekcji sygnałów oraz systemy wspomagania satelitarnego takie jak EGNOS, WAAS, MSAS i GAGAN. Ponadto, omówione zostaną techniki korekcji sygnałów, wpływ atmosfery na sygnały GNSS, z uwzględnieniem teorii synchronizacji czasu oraz technologii White Rabbit. Na końcu dokumentu dodano szczegółową analizę integracji GNSS z systemami inercjalnymi (INS), co pozwala na poprawę dokładności i niezawodności nawigacji.
W niniejszej pracy odwołujemy się do literatury [\cite{examplebook}].

% projekt/rozdzialy/sygnaly_i_czestotliwosci.tex
% project/chapters/signals_and_frequencies.tex
\section{Informacje o sygnałach i częstotliwościach GNSS}
\begin{itemize}
    \item \textbf{GPS}
    \begin{itemize}
        \item \textbf{Częstotliwości}: L1 (1575.42 MHz), L2 (1227.60 MHz), L5 (1176.45 MHz)
        \item \textbf{Sygnały}:
        \begin{itemize}
            \item L1: C/A (Coarse/Acquisition) dla cywilnych, P(Y) dla wojskowych
            \item L2: P(Y) dla wojskowych, L2C dla cywilnych
            \item L5: Nowy sygnał dla zastosowań cywilnych, oferujący lepszą precyzję
        \end{itemize}
    \end{itemize}

    \item \textbf{Galileo}
    \begin{itemize}
        \item \textbf{Częstotliwości}: E1 (1575.42 MHz), E5a (1176.45 MHz), E5b (1207.14 MHz), E6 (1278.75 MHz)
        \item \textbf{Sygnały}:
        \begin{itemize}
            \item E1: Open Service (OS), Public Regulated Service (PRS)
            \item E5a i E5b: Open Service (OS), Safety of Life (SoL)
            \item E6: Commercial Service (CS)
        \end{itemize}
    \end{itemize}

    \item \textbf{BeiDou}
    \begin{itemize}
        \item \textbf{Częstotliwości}: B1 (1561.098 MHz), B2 (1207.14 MHz), B3 (1268.52 MHz)
        \item \textbf{Sygnały}:
        \begin{itemize}
            \item B1: Open Service (OS)
            \item B2: Open Service (OS)
            \item B3: Restricted Service (RS)
        \end{itemize}
    \end{itemize}

    \item \textbf{GLONASS}
    \begin{itemize}
        \item \textbf{Częstotliwości}: L1 (1602 MHz + $k \cdot 0.5625$ MHz), L2 (1246 MHz + $k \cdot 0.4375$ MHz), gdzie \( k \) jest numerem kanału
        \item \textbf{Sygnały}:
        \begin{itemize}
            \item L1: Standard Precision (SP) dla cywilnych, High Precision (HP) dla wojskowych
            \item L2: HP dla wojskowych
        \end{itemize}
    \end{itemize}

    \item \textbf{QZSS}
    \begin{itemize}
        \item \textbf{Częstotliwości}: L1 (1575.42 MHz), L2 (1227.60 MHz), L5 (1176.45 MHz), L6 (1278.75 MHz)
        \item \textbf{Sygnały}:
        \begin{itemize}
            \item L1: C/A, SAIF
            \item L2: L2C
            \item L5: L5
            \item L6: LEX
        \end{itemize}
    \end{itemize}

    \item \textbf{IRNSS (NavIC)}
    \begin{itemize}
        \item \textbf{Częstotliwości}: L5 (1176.45 MHz), S (2492.028 MHz)
        \item \textbf{Sygnały}:
        \begin{itemize}
            \item L5: SPS (Standard Positioning Service)
            \item S: SPS
        \end{itemize}
    \end{itemize}

    \item \textbf{EGNOS}
    \begin{itemize}
        \item \textbf{Częstotliwości}: 1575.42 MHz (L1, GPS)
        \item \textbf{Sygnały}: EGNOS zapewnia korekty różnicowe dla sygnałów GPS i Galileo, zwiększając dokładność i niezawodność.
    \end{itemize}

    \item \textbf{WAAS}
    \begin{itemize}
        \item \textbf{Częstotliwości}: 1575.42 MHz (L1, GPS)
        \item \textbf{Sygnały}: WAAS zapewnia korekty różnicowe dla sygnałów GPS, zwiększając dokładność i niezawodność.
    \end{itemize}

    \item \textbf{MSAS}
    \begin{itemize}
        \item \textbf{Częstotliwości}: 1575.42 MHz (L1, GPS)
        \item \textbf{Sygnały}: MSAS zapewnia korekty różnicowe dla sygnałów GPS, zwiększając dokładność i niezawodność.
    \end{itemize}

    \item \textbf{GAGAN}
    \begin{itemize}
        \item \textbf{Częstotliwości}: 1575.42 MHz (L1, GPS), 1176.45 MHz (L5, GPS)
        \item \textbf{Sygnały}: GAGAN zapewnia korekty różnicowe dla sygnałów GPS, zwiększając dokładność i niezawodność na terenie Indii.
    \end{itemize}
\end{itemize}

% projekt/rozdzialy/qam.tex
% project/chapters/qam.tex
\section{Wzór Quadrature Amplitude Modulation}
Wzór \textbf{modulacji QAM (Quadrature Amplitude Modulation)}, można wyrazić jako:
\[
s_{T(i)}(t) = P_{T} \sum_{u=-\infty}^{\infty} \left[ d_I(i)(u) \cdot g_I(t - uT_b) + i \cdot d_Q(i)(u) \cdot g_Q(t - uT_b) \right]
\]

Gdzie:
\begin{itemize}
    \item \( P_{T} \) – Współczynnik skalujący dla sygnału (często związany z mocą sygnału).
    \item \( d_I(i)(u) \) – Ciąg symboli binarnych dla składowej in-phase (I) w \( u \)-tym symbolu.
    \item \( d_Q(i)(u) \) – Ciąg symboli binarnych dla składowej kwadratury (Q) w \( u \)-tym symbolu.
    \item \( g_I(t) \) – Funkcja odpowiadająca za kształt składowej in-phase (I), często nazywana funkcją odpowiedzi impulsowej.
    \item \( g_Q(t) \) – Funkcja odpowiadająca za kształt składowej kwadratury (Q), również nazywana funkcją odpowiedzi impulsowej.
    \item \( T_b \) – Okres symbolu (czas trwania jednego symbolu).
    \item \( i \) – Indeks kanału lub nośnej (w kontekście wielu nośnych w systemach GNSS).
\end{itemize}

\subsection{Wyprowadzenie Wzoru QAM}
W modulacji QAM, sygnał jest reprezentowany jako kombinacja dwóch sygnałów nośnych o tej samej częstotliwości, ale przesuniętych w fazie o 90 stopni (kwadraturowo). Sygnalizację można opisać wzorem:
\[
s(t) = I(t) \cos(2 \pi f_c t) - Q(t) \sin(2 \pi f_c t)
\]
gdzie \( I(t) \) i \( Q(t) \) są odpowiednio składowymi in-phase i kwadraturową, a \( f_c \) jest częstotliwością nośną.

\subsubsection{Transformacja Fouriera}
Analizując sygnał QAM w domenie częstotliwości za pomocą transformacji Fouriera:
\[
S(f) = \mathcal{F}\{s(t)\}
\]
Możemy uzyskać:
\[
S(f) = \frac{1}{2} \left[ I(f - f_c) + I(f + f_c) \right] - \frac{1}{2i} \left[ Q(f - f_c) - Q(f + f_c) \right]
\]

\subsubsection{Demodulacja Sygnałów QAM}
Demodulacja sygnałów QAM polega na ekstrakcji składowych \( I(t) \) i \( Q(t) \) poprzez mnożenie sygnału \( s(t) \) przez \( \cos(2 \pi f_c t) \) i \( \sin(2 \pi f_c t) \), a następnie filtrację dolnoprzepustową:
\[
r_I(t) = s(t) \cos(2 \pi f_c t)
\]
\[
r_Q(t) = s(t) \sin(2 \pi f_c t)
\]
\[
I(t) = \text{LPF}\{ r_I(t) \}
\]
\[
Q(t) = \text{LPF}\{ r_Q(t) \}
\]

% projekt/rozdzialy/korekcja_sygnalow.tex
% project/chapters/correction_techniques.tex
\section{Techniki Korekcji i Poprawki Sygnałów GNSS}

\subsection{DGPS (Differential GPS)}
DGPS jest techniką, która wykorzystuje stacje referencyjne umieszczone w znanych lokalizacjach do poprawy dokładności sygnałów GNSS. Stacje referencyjne odbierają sygnały GNSS i obliczają różnice między zmierzoną a znaną pozycją, generując poprawki dla użytkowników w okolicy.

\[
\Delta P = P_{\text{znane}} - P_{\text{zmierzone}}
\]

gdzie \( \Delta P \) to poprawka, \( P_{\text{znane}} \) to znana pozycja stacji referencyjnej, a \( P_{\text{zmierzone}} \) to zmierzona pozycja.

\subsection{RTK (Real-Time Kinematic)}
RTK jest techniką różnicową, która umożliwia osiągnięcie bardzo wysokiej dokładności (rzędu centymetrów) poprzez wykorzystywanie pomiarów fazowych sygnałów GNSS. RTK wymaga jednej lub więcej stacji referencyjnych, które przesyłają poprawki do użytkowników w czasie rzeczywistym.

\[
\Delta \phi = \phi_{\text{referencyjna}} - \phi_{\text{użytkownika}}
\]

gdzie \( \Delta \phi \) to poprawka fazy sygnału, \( \phi_{\text{referencyjna}} \) to faza sygnału odbieranego przez stację referencyjną, a \( \phi_{\text{użytkownika}} \) to faza sygnału odbieranego przez użytkownika.

\subsection{PPP (Precise Point Positioning)}
PPP jest techniką, która wykorzystuje precyzyjne efemerydy satelitów i dane o zegarach satelitarnych w połączeniu z zaawansowanymi modelami atmosferycznymi do osiągnięcia wysokiej dokładności bez potrzeby użycia stacji referencyjnych. PPP wymaga długiego czasu konwergencji, aby osiągnąć pełną dokładność.

\[
P_{\text{PPP}} = P_{\text{zmierzone}} + \Delta t_{\text{sat}} + \Delta t_{\text{odbiornik}} + \Delta \text{atmosfera}
\]

gdzie \( P_{\text{PPP}} \) to pozycja uzyskana za pomocą PPP, \( \Delta t_{\text{sat}} \) to błąd zegara satelity, \( \Delta t_{\text{odbiornik}} \) to błąd zegara odbiornika, a \( \Delta \text{atmosfera} \) to poprawka atmosferyczna.

% projekt/rozdzialy/wspomaganie_satelitarne.tex
% project/chapters/satellite_augmentation_systems.tex
\section{Systemy Wspomagania Satelitarnego}

\subsection{EGNOS (European Geostationary Navigation Overlay Service)}
EGNOS to system wspomagania satelitarnego, który dostarcza korekty dla sygnałów GPS i Galileo, zwiększając ich dokładność i niezawodność w Europie. EGNOS wykorzystuje sieć stacji referencyjnych, które monitorują sygnały GNSS i przesyłają poprawki do geostacjonarnych satelitów, które następnie retransmitują je do użytkowników.

\subsection{WAAS (Wide Area Augmentation System)}
WAAS to amerykański system wspomagania satelitarnego, który działa podobnie do EGNOS, dostarczając korekty dla sygnałów GPS w Ameryce Północnej.

\subsection{MSAS (Multi-functional Satellite Augmentation System)}
MSAS to japoński system wspomagania satelitarnego, który dostarcza korekty dla sygnałów GPS w regionie Azji Wschodniej.

\subsection{GAGAN (GPS Aided Geo Augmented Navigation)}
GAGAN to indyjski system wspomagania satelitarnego, który dostarcza korekty dla sygnałów GPS w Indiach i sąsiednich regionach.

% projekt/rozdzialy/algorytmy_przetwarzania.tex
% project/chapters/signal_processing_algorithms.tex
\section{Algorytmy Przetwarzania Sygnałów GNSS}

\subsection{Filtracja Kalmana}
Filtr Kalmana jest optymalnym estymatorem stanu układu dynamicznego, który minimalizuje średni kwadrat błędu estymacji. W kontekście GNSS, filtr Kalmana jest stosowany do estymacji pozycji, prędkości i innych parametrów ruchu.

\subsubsection{Równania Filtra Kalmana}
\begin{align}
\hat{x}_{k|k-1} &= F_{k-1} \hat{x}_{k-1|k-1} + B_{k-1} u_{k-1} \\
P_{k|k-1} &= F_{k-1} P_{k-1|k-1} F_{k-1}^T + Q_{k-1} \\
K_k &= P_{k|k-1} H_k^T (H_k P_{k|k-1} H_k^T + R_k)^{-1} \\
\hat{x}_{k|k} &= \hat{x}_{k|k-1} + K_k (z_k - H_k \hat{x}_{k|k-1}) \\
P_{k|k} &= (I - K_k H_k) P_{k|k-1}
\end{align}

gdzie:
\begin{itemize}
    \item \( \hat{x}_{k|k-1} \) - prognoza stanu w chwili \( k \) na podstawie informacji dostępnych do chwili \( k-1 \)
    \item \( P_{k|k-1} \) - macierz kowariancji błędu prognozy
    \item \( K_k \) - macierz wzmocnienia Kalmana
    \item \( \hat{x}_{k|k} \) - estymowany stan w chwili \( k \)
    \item \( P_{k|k} \) - macierz kowariancji błędu estymacji
    \item \( F_k \) - macierz przejścia stanu
    \item \( B_k \) - macierz kontrolna
    \item \( u_k \) - wektor wejściowy
    \item \( Q_k \) - macierz szumu procesu
    \item \( H_k \) - macierz obserwacji
    \item \( R_k \) - macierz szumu pomiarowego
    \item \( z_k \) - wektor pomiarowy
\end{itemize}

\subsection{Estymacja Pozycji}
Estymacja pozycji w GNSS polega na rozwiązaniu zestawu równań, które opisują odległości między odbiornikiem a satelitami. Najczęściej stosowaną metodą jest metoda najmniejszych kwadratów (LS).

\subsubsection{Równania Estymacji Pozycji}
\[
\mathbf{d} = \mathbf{H} \mathbf{x} + \mathbf{v}
\]
gdzie:
\begin{itemize}
    \item \( \mathbf{d} \) - wektor odległości satelita-odbiornik
    \item \( \mathbf{H} \) - macierz geometryczna
    \item \( \mathbf{x} \) - wektor nieznanych współrzędnych odbiornika i błędów zegara
    \item \( \mathbf{v} \) - wektor szumu pomiarowego
\end{itemize}

Rozwiązanie problemu najmniejszych kwadratów:
\[
\hat{\mathbf{x}} = (\mathbf{H}^T \mathbf{H})^{-1} \mathbf{H}^T \mathbf{d}
\]

\subsection{Synchronizacja Czasu}
Synchronizacja czasu jest kluczowym aspektem w GNSS, ponieważ dokładność pomiarów odległości zależy od precyzyjnej synchronizacji zegarów satelitów i odbiorników. Techniki synchronizacji obejmują zarówno bezpośrednie korekty zegara, jak i zaawansowane algorytmy estymacji błędów zegara.

\subsubsection{Równania Synchronizacji Czasu}
\[
\Delta t = t_{\text{sat}} - t_{\text{odbiornik}}
\]
gdzie \( \Delta t \) to różnica czasu między zegarem satelity \( t_{\text{sat}} \) a zegarem odbiornika \( t_{\text{odbiornik}} \).

% projekt/rozdzialy/synchronizacja_czasu.tex
% project/chapters/time_synchronization.tex
\section{Teoria Synchronizacji Czasu}
Synchronizacja czasu jest kluczowym aspektem w GNSS i obejmuje różne techniki i algorytmy mające na celu zapewnienie dokładnej synchronizacji zegarów satelitów i odbiorników.

\subsection{Podstawy Synchronizacji Czasu}
Synchronizacja czasu w GNSS opiera się na precyzyjnych pomiarach czasowych sygnałów satelitarnych. Każdy satelita GNSS posiada atomowy zegar, który dostarcza precyzyjne informacje czasowe. Odbiornik GNSS odbiera sygnały z kilku satelitów i porównuje czas nadania sygnału z czasem jego odbioru, co pozwala na obliczenie odległości do satelitów i synchronizację zegara odbiornika.

\subsection{Algorytmy Estymacji Błędów Zegara}
Algorytmy estymacji błędów zegara są stosowane do korekcji różnic czasowych między zegarami satelitów a zegarem odbiornika. Najczęściej stosowanym algorytmem jest filtr Kalmana, który estymuje błąd zegara na podstawie pomiarów różnic czasowych.

\[
\Delta t = t_{\text{sat}} - t_{\text{odbiornik}}
\]

gdzie \( \Delta t \) to różnica czasu między zegarem satelity \( t_{\text{sat}} \) a zegarem odbiornika \( t_{\text{odbiornik}} \).

\subsection{Technologia White Rabbit}
White Rabbit to zaawansowana technologia synchronizacji czasu, która łączy Ethernet z precyzyjnym protokołem synchronizacji czasu (PTP) oraz techniką cyfrowej kompensacji opóźnień (DDC). White Rabbit zapewnia synchronizację zegarów z dokładnością do sub-nanosekund, co jest kluczowe w aplikacjach wymagających ekstremalnej precyzji czasowej.

\subsubsection{Architektura White Rabbit}
Architektura White Rabbit opiera się na sieci Ethernet, w której węzły są wyposażone w zegary atomowe lub wysokiej jakości oscylatory. Protokół PTP jest używany do synchronizacji czasu między węzłami, a technika DDC kompensuje opóźnienia wprowadzane przez sprzęt sieciowy.

\subsubsection{Równania Synchronizacji White Rabbit}
Równania synchronizacji White Rabbit opierają się na modelu opóźnień i korekcji czasowych.

\[
t_{\text{wr}} = t_{\text{ptp}} - \Delta t_{\text{ddc}}
\]

gdzie:
\begin{itemize}
    \item \( t_{\text{wr}} \) - zsynchronizowany czas White Rabbit
    \item \( t_{\text{ptp}} \) - czas uzyskany z protokołu PTP
    \item \( \Delta t_{\text{ddc}} \) - kompensacja opóźnień wprowadzanych przez sprzęt (DDC)
\end{itemize}

White Rabbit znajduje zastosowanie w aplikacjach takich jak akceleratory cząstek, systemy telekomunikacyjne, finansowe systemy transakcyjne oraz sieci energetyczne, gdzie precyzyjna synchronizacja czasu jest krytyczna.

% projekt/rozdzialy/wplyw_atmosfery.tex
% project/chapters/atmospheric_effects.tex
\section{Wpływ Atmosfery na Sygnały GNSS}

\subsection{Wpływ Jonosfery}
Jonosfera jest warstwą atmosfery znajdującą się na wysokości od około 60 km do 1000 km nad powierzchnią Ziemi. Jest zjonizowana przez promieniowanie słoneczne, co powoduje, że ma znaczący wpływ na propagację sygnałów GNSS.

\subsubsection{Opóźnienie Jonosferyczne}
Opóźnienie jonosferyczne jest wynikiem refrakcji sygnałów GNSS w jonosferze. Jest ono odwrotnie proporcjonalne do kwadratu częstotliwości sygnału.

\[
\Delta t_{\text{iono}} = \frac{40.3}{f^2} \cdot \text{TEC}
\]

gdzie:
\begin{itemize}
    \item \( \Delta t_{\text{iono}} \) - opóźnienie jonosferyczne
    \item \( f \) - częstotliwość sygnału GNSS
    \item TEC - całkowita zawartość elektronów (Total Electron Content) w jonosferze
\end{itemize}

\subsubsection{Metody Kompensacji}
Istnieją różne metody kompensacji opóźnienia jonosferycznego, w tym:
\begin{itemize}
    \item \textbf{Model Klobuchara}: Empiryczny model, który dostarcza współczynniki do obliczania opóźnienia jonosferycznego.
    \item \textbf{Podwójne częstotliwości}: Wykorzystanie dwóch różnych częstotliwości sygnałów GNSS do obliczenia i skompensowania opóźnienia jonosferycznego.
\end{itemize}

\subsection{Wpływ Troposfery}
Troposfera jest najniższą warstwą atmosfery, sięgającą do około 10-15 km nad powierzchnią Ziemi. Zawiera większość pary wodnej i aerozoli, co wpływa na propagację sygnałów GNSS.

\subsubsection{Opóźnienie Troposferyczne}
Opóźnienie troposferyczne składa się z dwóch komponentów: suchego i mokrego. Komponent suchy zależy głównie od ciśnienia atmosferycznego, podczas gdy komponent mokry zależy od zawartości pary wodnej.

\[
\Delta t_{\text{tropo}} = \Delta t_{\text{dry}} + \Delta t_{\text{wet}}
\]

\subsubsection{Metody Kompensacji}
Metody kompensacji opóźnienia troposferycznego obejmują:
\begin{itemize}
    \item \textbf{Model Saastamoinena}: Empiryczny model, który uwzględnia ciśnienie, temperaturę i wilgotność do obliczenia opóźnienia troposferycznego.
    \item \textbf{Korekcje na podstawie obserwacji meteorologicznych}: Użycie danych meteorologicznych do poprawy dokładności modeli opóźnienia troposferycznego.
\end{itemize}

\subsection{Wpływ Warunków Lokalnych}
Warunki lokalne, takie jak topografia terenu, zabudowania i roślinność, mogą również wpływać na propagację sygnałów GNSS, powodując wielotorowość (multipath) i inne zakłócenia.

\subsubsection{Wielotorowość (Multipath)}
Wielotorowość jest efektem odbić sygnałów GNSS od powierzchni takich jak budynki, woda czy teren, co powoduje, że odbiornik odbiera sygnały z opóźnieniem.

\subsubsection{Metody Kompensacji}
Techniki redukcji efektów wielotorowości obejmują:
\begin{itemize}
    \item \textbf{Filtracja sygnałów}: Użycie zaawansowanych algorytmów filtracji w celu odrzucenia odbitych sygnałów.
    \item \textbf{Projektowanie anten}: Użycie anten o wysokiej kierunkowości i odporności na wielotorowość.
\end{itemize}

% projekt/rozdzialy/integracja_gnss_ins.tex
% project/chapters/gnss_ins_integration.tex
\section{Integracja GNSS z Inertial Navigation Systems (INS)}
Integracja GNSS z Inertial Navigation Systems (INS) pozwala na poprawę dokładności i niezawodności nawigacji poprzez połączenie zalet obu systemów. GNSS dostarcza dokładnych informacji o położeniu w długim okresie czasu, podczas gdy INS zapewnia wysoką dokładność w krótkim okresie, zwłaszcza w warunkach, gdzie sygnały GNSS mogą być zakłócone lub niedostępne.

\subsection{Podstawy Systemów Inercjalnych}
Systemy inercjalne wykorzystują akcelerometry i żyroskopy do mierzenia przyspieszeń i prędkości kątowych, które są następnie integrowane w celu uzyskania pozycji i orientacji. Podstawowe równania dla systemu inercjalnego można zapisać jako:

\[
\mathbf{v}(t) = \mathbf{v}(t_0) + \int_{t_0}^{t} \mathbf{a}(\tau) d\tau
\]
\[
\mathbf{r}(t) = \mathbf{r}(t_0) + \int_{t_0}^{t} \mathbf{v}(\tau) d\tau
\]

gdzie:
\begin{itemize}
    \item \( \mathbf{v}(t) \) - prędkość w chwili \( t \)
    \item \( \mathbf{a}(t) \) - przyspieszenie w chwili \( t \)
    \item \( \mathbf{r}(t) \) - pozycja w chwili \( t \)
    \item \( t_0 \) - czas początkowy
\end{itemize}

\subsection{Fuzja Danych GNSS i INS}
Integracja GNSS i INS polega na fuzji danych z obu systemów w celu uzyskania bardziej dokładnych i niezawodnych informacji o położeniu i orientacji. W praktyce najczęściej stosuje się Filtr Kalmana rozszerzony (EKF) do fuzji danych GNSS i INS. Podstawowe równania EKF są następujące:

\[
\hat{x}_{k|k-1} = f(\hat{x}_{k-1|k-1}, u_{k-1})
\]
\[
P_{k|k-1} = F_{k-1} P_{k-1|k-1} F_{k-1}^T + Q_{k-1}
\]
\[
K_k = P_{k|k-1} H_k^T (H_k P_{k|k-1} H_k^T + R_k)^{-1}
\]
\[
\hat{x}_{k|k} = \hat{x}_{k|k-1} + K_k (z_k - h(\hat{x}_{k|k-1}))
\]
\[
P_{k|k} = (I - K_k H_k) P_{k|k-1}
\]

gdzie:
\begin{itemize}
    \item \( f(\cdot) \) - nieliniowa funkcja przejścia stanu
    \item \( h(\cdot) \) - nieliniowa funkcja obserwacji
    \item \( F_{k-1} \) - macierz Jacobiana funkcji przejścia stanu
    \item \( H_k \) - macierz Jacobiana funkcji obserwacji
\end{itemize}

\subsection{Przykłady Implementacji Algorytmów GNSS i INS}
Poniżej przedstawiono przykładową implementację filtra Kalmana do integracji danych GNSS i INS w Pythonie:

\lstinputlisting[caption=Implementacja Filtra Kalmana do Integracji GNSS i INS w Pythonie]{ekf_gnss_ins.py}

% projekt/rozdzialy/podsumowanie.tex
% project/chapters/conclusion.tex
\section{Zakończenie}
W niniejszym dokumencie przedstawiono kluczowe metody analizy sygnałów GNSS, podkreślając ich znaczenie w różnych zastosowaniach technologicznych. Każda z tych metod wnosi istotny wkład w rozwój nowoczesnych systemów nawigacyjnych i komunikacyjnych. Dodatkowo omówiono techniki korekcji sygnałów GNSS, algorytmy przetwarzania oraz wpływ atmosfery na sygnały, co stanowi kompleksowy przegląd technologii GNSS na poziomie doktoranckim. Szczególna uwaga została poświęcona teorii synchronizacji czasu oraz technologii White Rabbit, które są kluczowe dla zapewnienia precyzyjnej synchronizacji w zaawansowanych aplikacjach GNSS. Dodatkowo, rozważano integrację GNSS z systemami inercjalnymi (INS), co pozwala na poprawę dokładności i niezawodności w różnorodnych warunkach operacyjnych.


%\chapter{Implementacja EKF w Pythonie}
%\chapter{Implementacja EKF w Pythonie}

Poniżej przedstawiono przykładową implementację filtra Kalmana do integracji GNSS i INS w Pythonie:

\begin{lstlisting}[caption={Implementacja EKF w Pythonie}, label={lst:EKF_Python}]
import numpy as np

class EKF:
    def __init__(self, f, h, F, H, Q, R, P, x0):
        self.f = f      # Funkcja przejścia stanu
        self.h = h      # Funkcja obserwacji
        self.F = F      # Jakobian funkcji przejścia stanu
        self.H = H      # Jakobian funkcji obserwacji
        self.Q = Q      # Macierz kowariancji szumu procesowego
        self.R = R      # Macierz kowariancji szumu obserwacji
        self.P = P      # Macierz kowariancji błędu estymacji
        self.x = x0     # Estymata stanu

    def predict(self, u):
        # Predykcja stanu i kowariancji stanu
        self.x = self.f(self.x, u)
        self.P = self.F(self.x, u) @ self.P @ self.F(self.x, u).T + self.Q

    def update(self, z):
        # Obliczanie wzmocnienia Kalmana
        y = z - self.h(self.x)
        S = self.H(self.x) @ self.P @ self.H(self.x).T + self.R
        K = self.P @ self.H(self.x).T @ np.linalg.inv(S)

        # Aktualizacja estymaty stanu i kowariancji błędu estymacji
        self.x = self.x + K @ y
        self.P = self.P - K @ self.H(self.x) @ self.P
\end{lstlisting}



\printbibliography

\end{document}
